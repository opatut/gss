\documentclass[a4paper,11pt]{article}

\newcommand{\authorinfo}{Carolin Konietzny, Paul Bienkowski, Julian Tobergte, Oliver Sengpiel, Lars Thoms}
\newcommand{\titleinfo}{GSS Abgabe 02}

% PREAMBLE ===============================================================

\usepackage[german,ngerman]{babel}
\usepackage[utf8]{inputenc}
\usepackage[T1]{fontenc}
\usepackage[top=1.3in, bottom=1in, left=1.0in, right=0.6in]{geometry}
\usepackage{lmodern}
\usepackage{amssymb}
\usepackage{mathtools}
\usepackage{amsmath}
\usepackage{enumerate}
\usepackage{pgfplots}
\usepackage{breqn}
\usepackage{tikz}
\usepackage{fancyhdr}
\usepackage{multicol}
\usepackage{xhfill}

\usetikzlibrary{calc}
\usetikzlibrary{patterns}

\author{\authorinfo}
\title{\titleinfo}
\date{\today}

\pagestyle{fancy}
\fancyhf{}
\fancyhead[L]{\authorinfo}
\fancyhead[R]{\titleinfo}
\fancyfoot[C]{\thepage}

\newcommand{\ditto}{$\vdots$}
\newcommand{\Frac}[2]{\frac{#1}{#2}}

\begin{document}
\maketitle


\begin{enumerate}
\item[\textbf{1}]
    \textbf{(Grundlagen von Betriebssystemen)}
    \begin{enumerate} 
        \item[a)]
        \begin{itemize}
            \item \textbf{Abstraktion}: Das Betriebssystem bietet ein ,,schönes'' (aufgeräumt, übersichtlich, verständlich, standardisiert) Interface zur Kommunikation mit der Hardware. Dies macht es Anwendungsentwicklern einfacher damit zu interagieren.
            \item \textbf{Resourcenmanagement}: Das Betriebssystem weist den Programmen Speicher und Rechenzeit zu. Es verwaltet Gerate und Dateien. 

        \end{itemize}
        \item[b)]
        \begin{itemize}
            \item \textbf{Abstraktion}: Verschiedene Hardwaremodule des gleichen Typs (z.B. USB-Sticks versch. Hersteller oder WLAN/Ethernetkarte als verschiedene Netzwerkadapter) können über die gleiche Schnittstelle angesprochen werden. Damit wird die Komplexität der eigentlichen Maschine versteckt.  
            \item \textbf{Resourcenmanagement}: Der Hauptspeicher wird verwaltet, den Programmen werden verschiedene Speicherbereiche zugewiesen. Die Festplatte wird in logische Segmente unterteilt, das Dateisystem verwaltet Dateien als logische, benannte Einheiten.
        \end{itemize}   
    \end{enumerate}
\item[\textbf{2}]
    \textbf{(Prozess und Threads)}


\item[\textbf{3}]
    \textbf{(n-Adressmaschine)}

    Wir dürfen keine Hilfsregister verwenden, benötigen aber in einer 2-Adress-Maschine mindestens 2 Register um Berechnungen durchzuführen. Daher bennen wir unsere Register R1 und R2, als Hilfsspeicherzelle verwenden wir H1.

    \begin{tabular}{lll|c|c|c|c}
        Befehl & & & $R_1$ & $R_2$ & $H_1$ & $Z$ \\ \hline
        LOAD & $a_1$ & $R_1$ & $a_1$ & & & \\
        LOAD & $a_2$ & $R_2$ & \ditto & $a_2$ & & \\
        ADD  & $R_1$ & $R_2$ & \ditto & $a_1 + a_2$ & & \\
        LOAD & $a_3$ & $R_1$ & $a_3$ & \ditto & & \\
        DIV  & $R_2$ & $R_1$ & $\Frac{a_1 + a_2}{a_3}$ & \ditto & & \\
        STORE& $R_1$ & $H_1$ & \ditto & \ditto & $\Frac{a_1 + a_2}{a_3}$ & \\
        LOAD & $b_1$ & $R_1$ & $b_1$ & \ditto & \ditto & \\
        LOAD & $b_2$ & $R_2$ & \ditto & $b_2$ & \ditto & \\
        ADD  & $R_1$ & $R_2$ & \ditto & $b_1 + b_2$ & \ditto & \\
        LOAD & $b_3$ & $R_1$ & $b_3$ & \ditto & \ditto & \\
        DIV  & $R_2$ & $R_1$ & $\Frac{b_1 + b_2}{b_3}$ & \ditto & \ditto & \\
        LOAD & $H_1$ & $R_2$ & \ditto & $\Frac{a_1 + a_2}{a_3}$ & \ditto & \\
        ADD  & $R_2$ & $R_1$ & $\Frac{a_1 + a_2}{a_3} + \Frac{b_1 + b_2}{b_3}$ & \ditto & \ditto & \\
        STORE& $R_1$ & $Z$   & \ditto & \ditto & \ditto & $\Frac{a_1 + a_2}{a_3} + \Frac{b_1 + b_2}{b_3}$
    \end{tabular}\\[10pt]

    \begin{tabular}{ll}
    Leseoperationen:& 7 \\
    Schreiboperationen:& 2 \\
    Rechenbefehle:& 5\\
    Berechnungszeit:& $(7+2)*20 + 5 = 185$ FLOP
    \end{tabular}


\end{enumerate}
\end{document}
