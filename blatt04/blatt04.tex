\documentclass[a4paper,11pt]{article}

\newcommand{\authorinfo}{Carolin Konietzny, Paul Bienkowski, Julian Tobergte, Oliver Sengpiel}
\newcommand{\titleinfo}{GSS Abgabe 04}

% PREAMBLE ===============================================================

\usepackage[german,ngerman]{babel}
\usepackage[utf8]{inputenc}
\usepackage[T1]{fontenc}
\usepackage[top=1.3in, bottom=1in, left=1.0in, right=0.6in]{geometry}
\usepackage{lmodern}
\usepackage{amssymb}
\usepackage{mathtools}
\usepackage{amsmath}
\usepackage{enumerate}
\usepackage{pgfplots}
\usepackage{breqn}
\usepackage{tikz}
\usepackage{fancyhdr}
\usepackage{multicol}
\usepackage{xhfill}
\usepackage{blockgraph} % IMPORTANT

\usetikzlibrary{calc}
\usetikzlibrary{patterns}
\usetikzlibrary{positioning}
\usetikzlibrary{shapes}
\usetikzlibrary{arrows}
\tikzset{main node/.style={ellipse,fill=blue!20,draw,minimum size=1cm,inner sep=0pt},
            }

\author{\authorinfo}
\title{\titleinfo}
\date{\today}

\pagestyle{fancy}
\fancyhf{}
\fancyhead[L]{\authorinfo}
\fancyhead[R]{\titleinfo}
\fancyfoot[C]{\thepage}

\newcommand{\ditto}{$\vdots$}
\newcommand{\Frac}[2]{\frac{#1}{#2}}

\begin{document}
\maketitle


\begin{enumerate}
\item[\textbf{1.1}]

\begin{itemize}
    \item[a)]

    \item[b)]
\end{itemize}

\item[\textbf{2}]

\begin{itemize}
    \item[a)]

    \item[b)]

    \item[c)] 
\end{itemize}

\item[\textbf{3}]

\begin{itemize}
    \item[a)]

    Zeitskala: t/5

    \begin{blockgraph}{34}{4}{0.4} % 170 / 5 = 34 cols
    % \bglabelxx erzeut Beschriftung der X-Achse an bestimmter Position
    \bglabelxx{0}
    \bglabelxx{5}
    \bglabelxx{10}
    \bglabelxx{15}
    \bglabelxx{20}
    \bglabelxx{25}
    \bglabelxx{30}
    \bglabelxx{34}

    % \bglabely erzeut eine Beschriftung der X-Achse an bestimmter Position
    \bglabely{3}{Periodendauer Z}
    \bglabely{2}{Periodendauer B}
    \bglabely{1}{Periodendauer M}
    \bglabely{0}{Berechnung}

    % \bgblock erzeugt Block innerhalb des Graphen. Parameter:
    %    Y-Position (z.B. CPU), optional
    %    Beginn auf der X-Achse
    %    Ende auf der X-Achse
    %    Beschriftung
    \bgblock[0]{0}{6}{$M$}
    \bgblock[0]{8}{5}{$B$}
    \bgblock[0]{12}{16}{$Z$}

    \bgblock[1]{0}{23}{$M$}
    \bgblock[1]{23}{46}{$M$} % 115 / 5 = 23
    \bgblock[2]{0}{8}{$B$}
    \bgblock[2]{8}{16}{$B$}
    \bgblock[2]{16}{24}{$B$}
    \bgblock[2]{24}{32}{$B$} % 40 / 5 = 8
    \bgblock[3]{0}{12}{$Z$}
    \bgblock[3]{12}{24}{$Z$}
    \bgblock[3]{24}{36}{$Z$} % 60 / 5 = 12



\end{blockgraph}
\end{itemize}

\end{enumerate}

\begin{blockgraph}{15}{1}{0.4}
    % \bglabelxx erzeut Beschriftung der X-Achse an bestimmter Position
    \bglabelxx{0}
    \bglabelxx{5}
    \bglabelxx{10}
    \bglabelxx{15}

    % \bgblock erzeugt Block innerhalb des Graphen. Parameter:
    %    Y-Position (z.B. CPU), optional
    %    Beginn auf der X-Achse
    %    Ende auf der X-Achse
    %    Beschriftung
    \bgblock{0}{4}{$P_1$}
    \bgblock{4}{7}{$P_2$}
    \bgblock{7}{12}{$P_3$}

\end{blockgraph}

\end{document}
