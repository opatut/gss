\documentclass[a4paper,11pt]{article}

\newcommand{\authorinfo}{Carolin Konietzny, Paul Bienkowski, Julian Tobergte, Oliver Sengpiel}
\newcommand{\titleinfo}{GSS Abgabe 06}

% PREAMBLE ===============================================================

\usepackage[german,ngerman]{babel}
\usepackage[utf8]{inputenc}
\usepackage[T1]{fontenc}
\usepackage[top=1.3in, bottom=1in, left=1.0in, right=0.6in]{geometry}
\usepackage{lmodern}
\usepackage{amssymb}
\usepackage{mathtools}
\usepackage{amsmath}
\usepackage{enumerate}
\usepackage{pgfplots}
\usepackage{breqn}
\usepackage{tikz}
\usepackage{fancyhdr}
\usepackage{multicol}
\usepackage{xhfill}
\usepackage{blockgraph} % IMPORTANT
\usepackage{listings}

\lstset{
    basicstyle=\footnotesize,
}

\usetikzlibrary{calc}
\usetikzlibrary{patterns}
\usetikzlibrary{positioning}
\usetikzlibrary{shapes}
\usetikzlibrary{arrows}
\tikzset{main node/.style={ellipse,fill=blue!20,draw,minimum size=1cm,inner sep=0pt},
            }

\author{\authorinfo}
\title{\titleinfo}
\date{\today}

\pagestyle{fancy}
\fancyhf{}
\fancyhead[L]{\authorinfo}
\fancyhead[R]{\titleinfo}
\fancyfoot[C]{\thepage}

\newcommand{\ditto}{$\vdots$}
\newcommand{\Frac}[2]{\frac{#1}{#2}}

\begin{document}
\maketitle


\begin{enumerate}
\item[\textbf{1.1}]

\begin{itemize}
    \item[a)] Wie groß ist eine Seite?

    Da unsere virtuellen Adressen 16 bit breit sind und wir 4 bit für die
    Adressierung der 16 Seiten in der Seitentabelle brauchen, haben wir ein
    Offset von \(2^{12}\) bit Breite.  Eine Seite enthält also 8 KB.

    \item[c)] Welche Physikalischen Adressen ergeben sich?
    \begin{enumerate}
        \item[i)] 0x0fe8
        \item[ii)] page fault
        \item[iii)] page fault
        \item[iv)] 0x8101
    \end{enumerate}

    \item[d)] Wahl einer optimalen Seitengröße:

    Größere pages führen dazu, dass große Bereiche des Speichers ungenutzt
    bleiben, kleinere pages führen zu größeren page tables und zu mehr over-
    head beim laden/speichern von pages.

    \item[e)] Optimale page size:

    Die optimale Größe einer Seite ergibt sich aus der Formel
    \[
        p = \sqrt{2 s e}
    \]
    mit \(s\) als Größe des Durchschnittlichen Prozesses und \(e\) als
    Wortbreite des Systems.

    In diesem Fall also \(\sqrt{2^{26}}\).
\end{itemize}

\item[\textbf{3.}]
\begin{itemize}
    \item[a)]
        \begin{verbatim}
Semaphor W, Mutex;
int NumberOfActiveReaders;

function Writer::processWriter() {
    while (NumberOfActiveReaders > 0) {
        wait();
    }

    W.lock();
    writeData();
    W.free();
}

function Reader::processReader() {
    Mutex.lock();
    NumberOfActiveReaders++;
    // only the first reader that appears needs to allocate the resource
    if (NumberOfActiveReaders == 1) {
        W.lock();
    }
    Mutex.free();

    readData();

    Mutex.lock();
    NumberOfActiveReaders--;
    // the last reader to finish frees the resource
    if (NumberOfActiveReaders == 0) {
        W.free();
    }
    Mutex.free();
}
        \end{verbatim}

    \item[b)]
        Reader werden hierbei bevorteiligt, denn wenn bereits mind. ein Reader läuft, kann ein weiterer sofort starten. Startet während dessen read-Vorgang
        wiederum ein weiterer Reader, bleibt die Resource geschützt. Somit können bei ,,ungünstigem'' Timing eine Reihe von Readern durchlaufen, ohne dass ein Writer, der u.U. schon länger auf Freigabe der Resource wartet, an die Reihe kommt.

\end{itemize}

\end{enumerate}

\end{document}
